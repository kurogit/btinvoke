\chapter{Projektbeschreibung}\label{chap:project}
%
Dieses Kapitel beschreibt die Ziele des Projektes. Es beantwortet die Frage: Was soll implementiert werden?

Das Hauptziel des Projektes ist die Implementierung eines Frameworks, das es erlaubt Berechnungen, oder sonstige Aktionen mit hoher Rechenleistung, auf ein zweites Android Gerät zu verteilen. Ein Gerät agiert als der Auslöser der Aktion. Auf ihm wird lediglich ein UI angezeigt, mit dem die Aktion gestartet werden kann. Das zweite Gerät wartet auf Anfragen des UI-Gerätes und führt diese aus. Danach sendet das berechnende Gerät eine Antwort zurück zum UI-Gerät. Dieses kann dem Benutzer dann das Ergebnis anzeigen.

% Beispielszenario
Ein Beispielszenario wäre folgendes: Man hat eine Smartwatch, sowie ein reguläres Smartphone. Über das UI auf der Smartwatch wird eine Aktion gestartet. Die dahinter liegenden Berechnungen werden allerdings nicht auf der Smartwatch ausgeführt, sondern auf dem Smartphone, das üblicherweise einen schnelleren Prozessor beinhaltet. Somit ist es möglich schneller das Ergebnis zu erhalten und auch Batterieladung auf der Smartwatch zu sparen.

% Was sind die Ziele
Das sekundäre Ziel des Projektes ist die Gewinnung von Erfahrung in Androidprogrammierung für den Autor.
%