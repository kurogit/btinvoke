\chapter{Einleitung}
%
Dieses Kapitel dient als Einleitung dieses Projektberichtes. Der erste Abschnitt beschreibt kurz, um was für ein Projekt es sich handelt. Der nächste Abschnitt geht auf Voraussetzungen und Anmerkungen zu dem Projekt ein. Im letzten Abschnitt wird der Aufbau dieses Dokumentes erläutert.
% Was ist das
%  Name des Frameworks: BTInvoke
\section{Projekt}
%
Dieses Projekt soll es ermöglichen, eine Java Methode, über Bluetooth, auf einem zweiten Androidgerät auszuführen. Es wird ein Framework erstellt. Dieses hat den Namen: \mbox{\emph{\textbf{BTInvoke}}}.
%
\section{Vorrausetzungen und Anmerkungen}
%
Die Implementierungen wurden in Eclipse Version 4.4.1 mit dem aktuell neusten ADK durchgeführt. Das \emph{Target API Level} war 19.

Um das erstellte Beispielapp korrekt benutzen zu können sind zwei Androidgeräte notwendig, die mindestens API Level 14 und Bluetooth unterstützten. Es wird empfohlen das App von Eclipse aus zu starten. Ansonsten muss ein APK erstellt und auf beiden Geräten installiert werden.
%
% Aufbau der Arbeit
\section{Aufbau des Dokumentes}
%
In \autoref{chap:project} wird das Projekt und die Ziele beschrieben. In \autoref{chap:impl} wird danach auf die Implementierung des Frameworks eingegangen. Probleme und Lösungen werden vorgestellt. \autoref{chap:example} enthält eine Beschreibung und ein Tutorial des erstellten Beispielapps. Im letzten \autoref{chap:fazit} wird ein Fazit zum Projekt gegeben und Verbesserungsvorschläge sowie mögliche Erweiterungen genannt.
%
%
